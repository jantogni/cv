\documentclass[11pt,a4paper]{moderncv}
\moderncvtheme[blue]{classic}
\usepackage[utf8]{inputenc}
\usepackage[scale=0.8]{geometry}
\AtBeginDocument{\recomputelengths}

% personal data
\firstname{\Huge Jonathan Antognini C.}
\familyname{}
\title{Estudiante de Ingeniería Civil Informática}
\address{}{Valpara\'iso, Chile}
\mobile{+56 9 630 833 96}
\email{jantogni@csrg.cl}
\photo[64pt]{image/me}
\begin{document}
\maketitle

\section{Educación}
\cventry{2007 a presente}
	{Estudiante de ingeniería civil informática}
	{Universidad Técnica Federico Santa María}
	{Valparaíso}
	{\emph{Egresado - Licenciado Ciencias de la Ingeniería - Memorista}}
	{Semestre 12/12}
\cventry{2003-2006}
	{Enseñanza media}
	{Liceo Neandro Schilling}
	{San Fernando, Libertador Bernando O'higgins}
	{}{}
%\vspace{-0.5cm}

\section{Experiencia}
\cventry{Práctica industrial 2011 Enero-Febrero}
	{Atacama large milimiter-submilimiter array}
	{El objetivo de la práctica fue, testear, desarrollar, documentar formas y procedimientos para el test automatizado de 
	interfaces gráficas de java.}
	{ALMA observatory}
	{Operation support facilities (OSF)}
	{}
\cventry{Práctica Profesional 2012 Agosto-Octubre}
	{Center for Technological Innovation in High Performance Computing - Valparaíso}
	{El objetivo de la práctica fue estudiar la forma de hacer algoritmos en Graphics processing unit (GPU) usando CUDA. Esto en el marco
	de la memoria, relacionado al área de High Frequency Trading.}
	{CTI-HPC}
	{Centro de innovación tecnológica - UTFSM}
	{}
\cventry{2009 octubre a 2011 Diciembre}
	{ALMA-UTFSM Ayudante de investigación}
	{Miembro del equipo de inteligencia artificial, trabajando en scheduling dinámico y clasificadores de imágenes para Very Large Telescope. 
		A partir del 2011 trabajando en ACS Windows Porting}
	{Computer Systems Research Group (CSRG)}
	{UTFSM}
	{}
\cventry{2011 marzo a 2011 Diciembre}
	{ALMA-UTFSM Líder de investigación}
	{Líder del equipo de investigación, gestionando y controlando los proyectos relacionados con inteligencia artificial e investigación.}
	{Computer Systems Research Group (CSRG)}
	{UTFSM}
	{}
\cventry{2012 marzo a presente}
	{ALMA-UTFSM Representante del grupo}
	{Representante del grupo a nivel administrativo y gestión de proyectos en conjunto con los encargados desde ALMA.}
	{Computer Systems Research Group (CSRG)}
	{UTFSM}
	{}
\cventry{2012 marzo a presente}
	{Coordinador CSRG}
	{Coordinador del Computer Systems Research Group, representando al grupo dentro de las iniciativas dentro de la universidad
		y controlando y apoyando las iniciativas que lo componen.}
	{Computer Systems Research Group (CSRG)}
	{UTFSM}
	{}
\cventry{2008 marzo a 2011 Octubre}
	{Ayudantía Matemáticas}
	{Matemáticas del bachillerato}
	{Departamento de matemática, UTFSM}
	{}{}
\cventry{2008 marzo a 2010 julio}
	{Ayudantía de programación}
	{Lenguaje Pascal}
	{Departamento de informática, UTFSM}
	{Valparaíso}
	{}
\cventry{2011 Octubre a 2012 Marzo}
	{Ayudantía de Computación Científica II}
	{Ayudante del laboratorio del ramo.}
	{Departamento de informática, UTFSM}
	{Valparaíso}
	{}
\cventry{2012 Abril a 2012 Julio}
	{Ayudantía de Computación Científica I}
	{Ayudante de cátedra}
	{Departamento de informática, UTFSM}
	{Valparaíso}
	{}
\cventry{2012 Agosto a 2012 Diciembre}
	{Ayudantía de Computación Científica II}
	{Ayudante coordinador del ramo. Participando como ayudante de cátedra y realizando 3 cátedras.}
	{Departamento de informática, UTFSM}
	{Valparaíso}
	{}
\cventry{2011 Octubre}
	{Redactor postulación FONDEF I+D}
	{Cómo líder de investigación de ALMA-UTFSM, se coordinó y redactó la postulación al FONDEF aprobado (D11I1060)}
	{Departamento de informática, UTFSM}
	{Valparaíso}
	{}

\section{Habilidades informáticas}
\cvlistitem{\textbf{Lenguajes}: C++, C\#, Python, Java, Prolog, Scheme, Pascal, Mathematica, Matlab, Qt.}{}
\cvlistitem{\textbf{DBA}: postgreSQL, MySQL.}{}
\cvlistitem{\textbf{Sistemas operativos}: (linux base) Redhat, Debian.}{}
\cvlistitem{\textbf{Sistemas operativos}: (Microsoft) Windows XP, 7}{}
\cvlistitem{\textbf{Web}: PHP, CSS, HTML.}{}  
\cvlistitem{\textbf{Móvil}: Android básico.}{}
\cvlistitem{\textbf{General}: Herramientas Microsoft Office, Latex, Git, Svn, Trac, Wiki.}

\section{Seminarios y eventos}
\cventry{2010}{XVIII Feria de Software}{Presentación del proyecto Asafis Mind}{UTFSM}{Valparaíso}{http://kratos.feriadesoftware.cl}
\cventry{2009}{EVIC}{VI Escuela de Verano Latino-americana en Inteligencia Computacional, III Encuentro de Verano Latino-americana en Robótica}{Universidad de Chile}{Santiago}{}
\cventry{2009}{ACS Workshop}{Presente en el trac Básico}{UTFSM}{Valparaíso}{}

\section{Idiomas}
\cvlanguage{Español}{Nativo}{Hablar, leer, escribir}
\cvlanguage{Inglés}{Medio}{Hablar, leer, escribir}

\section{Hobbies e intereses}
\cvlistitem{Competencias Proactividad, responsabilidad, vocación pedagógica}{}
\cvlistitem{Desarrollo de software, Métodos y modelos cuantitativos}{}
\cvlistitem{Deportes, Basketball, miembro de la selección de la universidad}{}
%\cvlistitem{Intento tocar algunos intrumentos musicales (guitarra, flauta, quena)}{}
%\cvlistitem{Películas, cine general, documentales, y cine alternativo}{}
%\renewcommand{\listitemsymbol}{-} 

\section{Premios}
\cvlistitem{Parte de la \emph{Lista de Honor} de la UTFSM, por la excelencia académica en los años 2008, 2009, 2010, 2012}
\cvlistitem{Primer lugar en excelencia académica en enseñanza media, Liceo Neandro Schilling.}

\section{Información Personal}
\cvcomputer{Nombre Completo}{Jonathan Andrés Antognini Cavieres}	{Run}{17.001.728-5}
\cvcomputer{Fecha de nacimiento}{14 Julio, 1988}			{Estado Civil}{Soltero}
\cvcomputer{Nacionalidad}{Chileno}          				{}{}

\end{document}
